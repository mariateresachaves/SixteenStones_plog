\documentclass[a4paper]{article}

%use the english line for english reports
%usepackage[english]{babel}
\usepackage[portuguese]{babel}
\usepackage[utf8]{inputenc}
\usepackage{indentfirst}
\usepackage{graphicx}
\usepackage{verbatim}
\usepackage[margin=0.8in]{geometry}
\usepackage{amsmath}

\begin{document}

\setlength{\textwidth}{16cm}
\setlength{\textheight}{22cm}

\title{\Huge\textbf{Sixteen Stones}\linebreak\linebreak\linebreak
\Large\textbf{Relatório Intercalar}\linebreak\linebreak
\linebreak\linebreak
\includegraphics[scale=0.1]{feup-logo.png}\linebreak\linebreak
\linebreak\linebreak
\Large{Mestrado Integrado em Engenharia Informática e Computação} \linebreak\linebreak
\Large{Programação em Lógica}\linebreak
}

\author{\textbf{Grupo Sixteen\_Stones\_3:}\\
Diogo Filipe Costa - ei11014 \\
Maria Teresa Chaves - up201306842 \\
\linebreak\linebreak \\
 \\ Faculdade de Engenharia da Universidade do Porto \\ Rua Roberto Frias, s\/n, 4200-465 Porto, Portugal \linebreak\linebreak\linebreak
\linebreak\linebreak\vspace{1cm}}

\maketitle
\thispagestyle{empty}

%----------------------------------------------------------------------------------------------------------------------------------------------------------------------
%----------------------------------------------------------------------------------------------------------------------------------------------------------------------

\newpage

%Todas as figuras devem ser referidas no texto. %\ref{fig:codigoFigura}
%
%%Exemplo de código para inserção de figuras
%%\begin{figure}[h!]
%%\begin{center}
%%escolher entre uma das seguintes três linhas:
%%\includegraphics[height=20cm,width=15cm]{path relativo da imagem}
%%\includegraphics[scale=0.5]{path relativo da imagem}
%%\includegraphics{path relativo da imagem}
%%\caption{legenda da figura}
%%\label{fig:codigoFigura}
%%\end{center}
%%\end{figure}
%
%
%\textit{Para escrever em itálico}
%\textbf{Para escrever em negrito}
%Para escrever em letra normal
%``Para escrever texto entre aspas''
%
%Para fazer parágrafo, deixar uma linha em branco.
%
%Como fazer bullet points:
%\begin{itemize}
	%\item Item1
	%\item Item2
%\end{itemize}
%
%Como enumerar itens:
%\begin{enumerate}
	%\item Item 1
	%\item Item 2
%\end{enumerate}
%
%\begin{quote}``Isto é uma citação''\end{quote}


%----------------------------------------------------------------------------------------------------------------------------------------------------------------------

\tableofcontents

\newpage

%----------------------------------------------------------------------------------------------------------------------------------------------------------------------

\section{O Jogo Sixteen Stones}

Sixteen Stones é o nome de um jogo de tabuleiro abstrato \footnote{Jogo abstrato - jogo de estratégia que tenta minimizar a sorte e não possui tema.} de forma quadrada, jogado normalmente num tabuleiro de 5x5, para dois jogadores. Inicialmente, cada jogador tem 8 pedras vermelhas ou azuis, colocando-as de forma alternada em qualquer uma das células livres, sendo que o jogador vermelho é o primeiro a jogar. Assim que todas as pedras estejam no tabuleiro, o jogo começa. O jogo termina quando o jogador vencedor consegue reduzir para um o número de pedras em jogo do adversário.

\subsection{Regras}

No início do jogo distribui-se 8 pedras (vermelhas ou azuis) para cada jogador. De forma alternada os jogadores devem colocá-las em qualquer uma das células livres, começando as pedras vermelhas.

\begin{figure}[!htb]
\centering
\includegraphics[scale=0.3]{board_prep.png} 
\caption{Preparação do tabuleiro.}
\end{figure}

Quando as pedras estiverem todas colocadas no tabuleiro, o jogador vermelho começa a jogar e alterna de turno com o jogador adversário.

\subsubsection{Jogadas possíveis}

Um jogador pode no seu respetivo turno realizar cada uma das seguintes jogadas: \textit{Push}, \textit{Move} e \textit{Sacrifice}.
No primeiro turno de cada jogador, deve ser feito um \textit{push} ou um \textit{move}, mas nunca ambos.

\begin{figure}[!htb]
\centering
\includegraphics[scale=0.3]{push2.png}
\caption{Jogador vermelha realiza um \textit{move} no seu primeiro turno.}
\end{figure}

\subsubsection{\textit{Push}}

Para realizar um  \textit{Push} é necessário que o jogador tenha mais pedras nessa linha que o adversário. Assim, se após um  \textit{Push}, a pedra do adversário é "empurrada" para fora do tabuleiro, então vai para o "banco" do respetivo jogador. As pedras que são "empurradas" apenas se movem uma célula na direção do \textit{Push}, que pode ser realizado em qualquer direção.

\begin{figure}[!htb]
\centering
\includegraphics[scale=0.3]{push.png} 
\caption{Direções possíveis de um \textit{Push} e representação do "banco".}
\end{figure}

\begin{figure}[!htb]
\centering
\includegraphics[scale=0.3]{push.png} 
\caption{Exemplo de \textit{Push} sem que a pedra adversária seja "empurrada" para fora do tabuleiro.}
\end{figure}

O número de pedras máximo (PE) que um jogador pode "empurrar" é igual:

\begin{equation}
PE = P - 1 \text{\hspace{7mm}(sendo P o número das suas pedras nessa linha)}
\end{equation}

Isto é, por exemplo, caso numa linha o jogador azul tenha três pedras, então pode realizar um  \textit{Push} a uma pedra ou a duas do adversário. Mas caso tenha duas apenas pode realizar um  \textit{Push} a uma pedra vermelha.

\begin{figure}[!htb]
\centering
\includegraphics[scale=0.3]{push2.png}
\caption{Push com três pedras.}
\end{figure}

Além das regras referidas anteriormente, um jogador não pode realizar um  \textit{Push} nas suas próprias pedras.

\begin{figure}[!htb]
\centering
\includegraphics[scale=0.3]{push2.png} 
\caption{Push inválido por realizar um \textit{Push} nas suas próprias pedras.}
\end{figure}

Por fim, para realizar um \textit{Push} é necessário que exista uma pedra para ser empurrada. Por exemplo caso numa linha o jogador azul tenha 3 pedras, mas nessa linha o jogador vermelho não tem pedras, então é realizar um  \textit{Push} é uma jogada inválida.

\begin{figure}[!htb]
\centering
\includegraphics[scale=0.3]{push2.png} 
\caption{Push inválido por falta de pedras do adversário.}
\end{figure}

\subsubsection{\textit{Move}}

É possível realizar um  \textit{Move} para uma célula vazia em qualquer direção. Se o jogador tentar realizar um \textit{Move} para uma célula ocupada, então é considerado uma jogada inválida.

\begin{figure}[!htb]
\centering
\includegraphics[scale=0.3]{push2.png} 
\caption{Exemplo de um \textit{Move} válido (a) e um \textit{Move} inválido (b).}
\end{figure}

Após um \textit{Move} se alguma das pedras adversárias ficar cercada\footnote{Uma pedra fica cercada quando numa das direções existem pedras adversárias nos dois sentidos.}, então esta é capturada e substituída por uma pedra do "banco" do jogador que a capturou.

\begin{figure}[!htb]
\centering
\includegraphics[scale=0.3]{push2.png} 
\caption{Exemplo de uma pedra capturada após um \textit{Move}.}
\end{figure}

\newpage

Se um jogador realizar um \textit{Move} e a sua pedra ficar voluntariamente cercada, esta não é capturada.

\begin{figure}[!htb]
\centering
\includegraphics[scale=0.3]{push2.png} 
\caption{Exemplo de uma pedra que após um \textit{Move} não é capturada.}
\end{figure}

\subsubsection{\textit{Sacrifice}}

Um jogador pode sacrificar uma pedra do seu "banco", permanentemente, para poder realizar um \textit{Push} ou um \textit{Move} adicional.

\begin{figure}[!htb]
\centering
\includegraphics[scale=0.3]{push2.png} 
\caption{Jogador azul realiza um \textit{Sacrifice} e depois um \textit{Push}}
\end{figure}

\newpage

\begin{itemize}
	%\item Os jogadores têm turnos para fazer cada jogada
	%\item Cada jogador pode no seu respetivo realizar cada uma das seguintes jogadas:
			%\begin{itemize}
				%\item[] \textbf{\textit{Push}}
				%\begin{itemize}
					%\item Para realizar um \textit{Push} é necessário que o jogador tenha mais peças nessa linha que o adversário
					%\item Quando uma peça é empurrada para fora do tabuleiro, vai para o "banco" do respetivo jogador
					%\item As peças que são empurradas, apenas se movem uma "casa" na direção do \textit{Push}
					%\item É possível realizar um \textit{Push} em qualquer direção
					%\item O número de peças máximo que um jogador pode empurrar é igual ao número das suas peças nessa linha menos um
					%\item Um jogador não pode realizar um \textit{Push} nas suas próprias peças
					%\item Para realizar um \textit{Push} é necessário que exista uma peça para ser empurrada
				%\end{itemize}
				%\item[] \textbf{\textit{Move}}
				%\begin{itemize}
					%\item É possível realizar um \textit{move} para qualquer "casa" desde que esta esteja vazia
					%\item Uma peça pode ser movida em qualquer direção nas "casas" vizinhas
					%\item Se, após um \textit{move}, alguma peça ficar cercada por peças adversárias, então esta é capturada.
					%\item Uma peça capturada é substituída por uma peça do "banco" do jogador que a capturou
					%\item Num \textit{move} em que a peça fica voluntariamente cercada, esta não é capturada
				%\end{itemize}
				%\item[] \textbf{\textit{Sacrifice}}
				%\begin{itemize}
					%\item Um jogador pode sacrificar uma peça do seu "banco", permanentemente, para poder realizar mais um \textit{push} ou um \textit{move} adicional
				%\end{itemize}
			%\end{itemize}
	\item No primeiro turno de cada jogador, deve ser feito um \textit{push} ou um \textit{move}, mas nunca ambos
	\item Se após um jogador realizar um \textit{Push}, uma das suas peças movidas não ocupa uma "casa" anteriormente ocupada por outra das suas peças,e esta cercar uma peça adversária, então esta é capturada
	\item Se um jogador empurrar uma peça adversária para uma posição de cerco, então esta é capturada
	\item Quando um jogador após uma captura substitui a peça capturada por uma do seu banco, esta não pode ser utilizada para capturar outra peça enquanto não for movida
	\item Se um jogador não tem peças no banco então este não pode capturar peças adversárias
\end{itemize}

Descrever detalhadamente o jogo, a sua história e, principalmente, as suas regras.
Devem ser incluidas imagens apropriadas para explicar o funcionamento do jogo.
Devem ser incluidas as fontes de informação (e.g. URLs em rodapé).


%----------------------------------------------------------------------------------------------------------------------------------------------------------------------
\section{Representação do Estado do Jogo}

Descrever a forma de representação do estado do tabuleiro (tipicamente uma lista de listas), com exemplificação em Prolog de posições iniciais do jogo, posições intermédias e finais, acompanhadas de imagens ilustrativas.


%----------------------------------------------------------------------------------------------------------------------------------------------------------------------
\section{Visualização do Tabuleiro}

Descrever a forma de visualização do tabuleiro em modo de texto e o(s) predicado(s) Prolog construídos para o efeito.
Deve ser incluída pelo menos uma imagem correspondente ao output produzido pelo predicado de visualização.


%----------------------------------------------------------------------------------------------------------------------------------------------------------------------
\section{Movimentos}

Elencar os movimentos (tipos de jogadas) possíveis e definir os cabeçalhos dos predicados que serão utilizados (ainda não precisam de estar implementados).

\section*{Bibliografia}

%https://pt.wikipedia.org/wiki/Jogo\_de\_estrat\%C3\%A9gia\_abstrato

\end{document}
